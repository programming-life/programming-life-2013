\documentclass[10pt,a4paper,notitlepage]{article}
\usepackage[utf8]{inputenc}
\usepackage{amsmath}
\usepackage{amsfonts}
\usepackage{amssymb}
\usepackage{graphicx}
\usepackage{fullpage}
\begin{document}
By combining \textbf{naturally occurring genetic components} in unique ways, it has become possible to \textbf{artificially} engineer \textbf{genetic networks} that possess sophisticated functional capabilities.

\textbf{Transcriptional control} operates at the level of \textbf{mRNA synthesis} through the use of \textbf{inducible transcriptional activators} and \textbf{repressors} that are capable of binding naturally occurring or specifically engineered \textbf{promoters}.
\textbf{Prokaryotic gene control systems} generally use inducible repressors and activators drawn from well-documented \textbf{genetic operons} such as the lac operon of Escherichia coli.
\textbf{Bacterial response regulators} also form the basis of synthetic \textbf{eukaryotic} gene regulation systems, although given transcriptional differences they require adaptation.

In considering the \textbf{design} of a synthetic genetic network for a \textbf{biological application} it is useful to imagine what kind of \textbf{functions} one might wish to create. 
Some applications may benefit from a \textbf{mechanism} that ensures a network produces a \textbf{consistent} and \textbf{stable response}.
For other applications, one may require a system that produces \textbf{more than one discrete expression state}.

To produce a \textbf{unified} and \textbf{consistent} \textbf{outcome} a biological process must be \textbf{capable of withstanding} a certain degree of \textbf{variation} and \textbf{difference}.
A key development in our understanding of how stability is maintained was through the discovery of \textbf{autoregulatory feedback loops} in which proteins, directly or indirectly, influence their own production.
An \textbf{autofeedback mechanism} can either be \textbf{negative}, in which a protein \textbf{inhibits} its own production, or \textbf{positive}, in which a protein \textbf{stimulates} its own production.

The \textbf{expression output} of many cell-based regulatory networks is often a \textbf{logic} response generated by one or more \textbf{input signals}.
Their \textbf{output} is either \textbf{ON} or \textbf{OFF} across a wide range of inducer concentrations, except for a small concentration window where transitions between the two states occur.
By utilizing several \textbf{compatible heterologous gene control systems}, it has been possible to design a range of \textbf{eukaryotic logic circuits} that follow strict \textbf{Boolean logic} in their integration of two input signals.

To \textbf{detect} weak transcriptional responses that, despite being difficult to detect \textit{in vivo}, are often involved in \textbf{regulatory functions} where only \textbf{trace amounts} of a gene product are required. 
In typical transcriptional studies aimed at \textbf{determining} the \textbf{conditions} under which a \textbf{promoter} is activated, a \textbf{reporter gene} is placed \textbf{downstream} of the promoter and assayed under varying conditions. 
However, where the promoter \textbf{response is weak} it is often \textbf{not possible} to discern any kind of activity. 
By placing a \textbf{repressor} cascade \textbf{downstream} of the promoter it was possible to \textbf{amplify} an otherwise undetectable promoter response.

The key requirements for a \textbf{band-detection network} (responding to an inducer within a given concentration range) are the design of modular components that enable the detection of a \textbf{low-threshold}, a \textbf{high threshold}, and a way of \textbf{integrating the two thresholds}.

The \textbf{pulse-generating network} produces output when a \textbf{threshold} concentration is reached, and then through a \textbf{feedforward mechanism} shuts down reporter expression regardless of whether the concentration continues to rise or fall.
Like the \textbf{band-detection network} the \textbf{pulse-generating network} provides important insights into how pulse-generating behavior could occur in \textbf{natural systems}.
\end{document}
